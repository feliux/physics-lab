Las fuerzas de cohesión molecular dentro de un volumen de líquido se cancelan mutuamente debido a la interacción de todas las partículas por igual y en todas las direcciones (aproximación adiabática). Sin embargo la situación difiere para las partículas que se encuentran en la superficie del fluido, pues se encuentran sometidas a una fuerza de cohesión únicamente por el lado orientado hacia el interior del volumen. Esta fuerza es responsable de la tensión superficial $\sigma$ que hace que la superficie de los líquidos se comporte como una membrana elástica.

\begin{equation}
	\sigma = \frac{F}{L} \label{eq_tension}
\end{equation}

A medida que la temperatura aumenta, la distancia media entre moléculas se incrementa y por tanto las fuerzas de cohesión disminuyen. A nivel macroscópico, este hecho se manifiesta como una disminución de la tensión superficial con la temperatura, hasta llegar a un punto crítico dado por la transición líquido-vapor. Loránd Eötvös observó que este comportamiento era aproximadamente lineal, y que podía describirse por

\begin{equation}
	\sigma V_{m}^{2/3} = k \left( T_k - T \right) \label{eq_eotvos}
\end{equation}

Donde $V_m$ representa el volumen molar del líquido, $T_k$ es una temperatura próxima a la temperatura crítica y $k = 2.1 \cdot 10^{-7}JK^{-1}mol^{-2/3}$ se conoce como constante de Eötvös.