En este experimento hemos determinado la tensión superficial del agua mediante el método de Du Nouy. Nuestro resultado ha sido notablemente inferior al valor teórico para el agua destilada. Hemos aprovechado éste cálculo para determinar la constante de Eötvös $k$ y la temperatura crítica $T_k$, siendo nuestros valores experimentales superior e inferior, respectivamente, a aquellos publicados en la literatura.

Estas diferencias pueden tener su origen en la toma de datos y el dispositivo experimental disponible en el laboratorio. En efecto, resulta muy complicado determinar los valores exactos de temperatura y, sobretodo, fuerza ejercida sobre el anillo. Contamos con una balanza de torsión analógica sobre la cual resulta difícil recoger el instante concreto en el cual se produce la separación del anillo de la superficie del agua. Además, no es imprescindible hacer una recalibración constante de la balanza entre medidas, lo cual induce errores aleatorios a lo largo de nuestro experimento y que, indudablemente, se trasladan en nuestros cálculos. Por ejemplo, en la figura (\ref{figure_eotvos}) vemos un ajuste muy pobre, $R^2 = 0.93$, que nos induce a pensar que tuvimos que repetir nuestro experimento para incluir más datos.