Contamos con un dispositivo experimental compuesto por un tubo capilar de vidrio de longitud $ L = 25.0 \pm 0.1$cm y radio $r^2 = (2.96 \pm 0.14)\cdot 10^{-6}m^2$, un tubo vertical de metacrilato graduado, manguera, balanza, vasos precipitados y cronómetro.

El capilar de vidrio se encuentra insertado horizontalmente al tubo vertical, el cual rellenaremos de agua hasta una distancia $h$, con la intención de que ésta circule hacia el capilar y sea depositada en un vaso precipitados. Mantendremos el nivel de agua dentro del tubo vertical gracias a una manguera conectada a un grifo que se encontrará abierto hasta que iniciemos el experimento. Al cerrar la manivela, y gracias al vaso de precipitados, registraremos la cantidad de masa de agua que desciende por la columna vertical para circular por el capilar durante un tiempo $t$. Así pues, con las medidas de cantidad de agua recogida, la altura de la columnta de agua y el tiempo podremos calcular las magnitudes objeto de esta práctica.

El radio del tubo capilar se ha calculado mediante la fórmula $r^2 = \frac{M}{\pi \rho L}$ donde $m$ es la cantidad de masa de agua que rellena el tubo capilar y $\rho$ la densidad del agua. Se ha optado por reflejar el valor $r^2$ en vez de $r$ dado que nuestros futuros cálculos conllevan términos cuadráticos en $r$.

Todas aquellas medidas que tengan incertidumbre asociada será explícitamente mencionado. Para aquellas magnitudes indirectas se calculará el error asociado según la fórmula habitual

\begin{equation}
	\Delta{A} = |\frac{\partial A}{\partial \alpha_i}|\Delta \alpha_i
\end{equation}

Todos los resultados experimentales, debidamente redondeados y presentados, se encuentran en la tabla (\ref{table_exp}) al final del informe.