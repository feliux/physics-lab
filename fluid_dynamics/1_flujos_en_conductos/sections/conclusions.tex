Se ha determinado la viscosidad del agua de manera experimental mediante el estudio de un caudal que fluye a través de un tubo capilar. Se ha obtenido un valor tres veces superior al valor teórico con un error relativo de entorno al 13\%, diferencias significativas debido a que se ha utilizado agua corriente que difiere en densidad frente al agua ideal (valor que hemos utilizado para nuestros cálculos) y posiblemente también debido a una diferencia de temperaturas para las cuales comparamos la viscosidad. Unido a esto, no podemos obviar la disposición experimental y posibles errores sistemáticos (no considerados) y aleatorios que podrían haber causado un desbarajuste en nuestros cálculos.

Sabiendo que todos estos errores se trasladarían a nuestros sucesivos cálculos, hemos caracterizado el régimen del fluido mediante el número de Reynolds a través de la velocidad media del fluido dentro del tubo capilar. Observamos pues el comportamiento laminar del fluido y su correspondiente paso a un régimen de semiturbulento. No hemos visto un flujo claramente turbulento, pues se exige un número de Reynolds $Re>3000$, valor que se consigue en circunstancias de baja viscosidad, altas velocidades o tuberías de gran sección. Aún así vemos que los valores obtenidos están en concordancia con la teoría expuesta en la introducción de este experimento.

Finalmante y gracias a los valores del número de Reynolds hemos calculado el coeficiente $\lambda$, para luego hacer un estudio en base al régimen del fluido mediante los diagramas de Moody, confirmando visualmente el punto crítico de transición aunque sin satisfacer la condición dada por Karmann-Prandtl en la ecuación (\ref{eq_karmann_prandtl}).

Dicho esto, podemos concluir un satisfactorio estudio cualitativo del fenómeno estudiado, más no cuantitativo, pues los sucesivos errores, su consecuente propagación además de las condiciones de laboratorio nos impide afirmar que tengamos unos buenos resultados para los parámetros del agua ideal.