En esta práctica se ha hecho patente la importancia de las simulaciones computacionales a la hora de estudiar sistemas físicos de muchas partículas. Así, gracias a la aplicación del algoritmo de Metrópolis se ha estudiado el potencial de Lennard-Jones que ejercen entre sí un conjunto de 200 átomos de argón. Variando la densidad de átomos así como la temperatura global del sistema se ha hecho latente el comportamiento de estas partículas cuando el material sufre una transición de estado sólido a estado gaseoso, verificando las posiciones de los átomos según el potencial de Lennard-Jones (a través de la función de distribución radial RDF) así como la distribución de energía cinética cuando aumenta la temperatura.

Hemos analizado un caso particular cuando la densidad de átomos es incluso mayor que aquella para la cual el potencial de Lennard-Jones es cero, para concluir que, dadas las circunstancias y mediante un modelos de esferas duras, resulta muy complicado que el material sufra una transición de estado al aumentar la temperatura, y por lo tanto, la energía cinética.

Aunque nuestros resultados fueron satisfactorios en el sentido de que corresponden con lo esperado en el ámbito del estudio de la materia condensada, hay que recalcar que nuestras simulaciones no son del todo realistas. Efectivamente, nuestro modelos de átomos está basado en esferas duras, incluso son representados por puntos cuando la densidad es bastante baja, lo cual nos lleva a concluir necesariamente que la densidad es un parámetro de escala que actúa como si hicieramos zoom en el plano bidimensional. Además, hemos analizado experimentalmente la actuación de un modelo de potencial que es experimental en sí mismo al depender del valor de $\sigma$ y $\epsilon$. Por lo tanto el estudio de simulaciones más realistas conllevaría incluir formulaciones teóricas o incluso otros componentes más elementales que expliquen el comportamiento de la física del sistema (teoría de orbitales, mecánica cuántica, etc).