\subsection{Efecto Hall}

Se conoce como efecto Hall a la aparición de un campo eléctrico en el interior de un conductor por el que circula una corriente $I$ en presencia de un campo magnético $B$. Las cargas que circulan por el conductor están sometidas a la fuerza de Lorentz

\begin{equation}
	\overrightarrow{F} = q(\overrightarrow{v}x\overrightarrow{B})
\end{equation}

Con lo cual aparece una fuerza magnética en los portadores de carga que los reagrupa a ambos lados del conductor de anchura $d$, apareciendo así una diferencia de pontencial en el conductor que origina un campo eléctrico perpendicular al campo magnético. Este campo eléctrico es el denominado campo Hall $E_H$, y ligada a él aparece la tensión Hall $V_H$ según la fórmula

\begin{equation}
	V_H = R_H\frac{IB}{d}
\end{equation}

Donde la constante $R_H$ se conoce como constante de Hall y equivale a

\begin{equation}
	R_H = \frac{1}{nq}
\end{equation}

Siendo $n$ la densidad de portadores de carga y $q$ la carga correspondiente. Además, según \cite{movilidad}, dicha constante se puede relacionar con la conductividad $\sigma$ y con la movilidad $\mu_H$ mediante la ecuación

\begin{equation}
	\mu = R_H \sigma
\end{equation}

\subsection{Efecto Hall y magnetorresistencia}

La magnetorresistencia, descubierta por William Thomson en 1857, es la propiedad que tienen algunos materiales de variar su resistencia eléctrica cuando son sometidos a un campo magnético. Este fenómeno puede relacionarse con el efecto Hall a través de la Ley de Ohm.

\begin{equation}
	R = \frac{V}{I}
\end{equation}

\subsection{Efecto Hall y temperatura}

Un semiconductor puede clasificarse según su conductividad en intrínseco y extrínseco. En los semiconductores intrínsecos los agentes conductores son los electrones y huecos que el material es capaz de generar térmicamente. El p-Germanium es un semiconductor extrínseco ya que es un semiconductor (intrínseco) dopado con materiales que lo proveen de un exceso de huecos para favorecer la conducción a bajas temperaturas. Partiendo del modelo clásico, la conductividad total será la suma de las contribuciones individuales de cada tipo de portador de carga libre, según la expresión

\begin{equation}
	\sigma = n_p \mu_p + n_h \mu_h
\end{equation}

Siendo $n_{p,h}$ la densidad de portadores y $\mu{p,h}$ la movilidad (el subíndice se refiere a portadores y huecos). Aplicando la estadística de Fermi-Dirac a la anterior ecuación y teniendo en cuenta que ambas poblaciones siguen la ley de acción de masas o del equilibrio $n_p n_h = n_i^{2}$ tenemos que se cumple

\begin{equation}
	\sigma = \sigma_0 e^{\frac{-E_g}{2K_bT}}
\end{equation}

Donde $\sigma_0$ es un prefactor que depende de la movilidad y la densidad de estados efectivos en las bandas de conducción y valencia, $E_g$ representa la energía del intervalo prohibido, $K_b$ es la constante de Boltzmann y $T$ la temperatura.