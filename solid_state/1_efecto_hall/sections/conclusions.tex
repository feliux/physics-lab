En esta práctica se ha estudiado el efecto Hall en una muestra de p-Germanium. Hemos verificado la relación lineal a través del coeficiente Hall $R_H$ entre el voltaje Hall y la intensidad de corriente por un lado y el campo magnético por otro. Gracias a este análisis hemos calculado la densidad de portadores de la muestra así como la correspondiente conductividad y movilidad de los portadores dentro del material.

También hemos comprobado experimentalmente la aparición del fenómeno conocido como magnetorresistencia, siendo el p-Germanium un material que aumenta su resistencia al someterlo a un campo magnético.

Se ha hecho un análisis semiclásico de la conductivdad al hacer variar la temperatura de la lámina, comprobándose así la teoría estadística de Fermi-Dirac para semiconductores dopados, es decir, inversión de voltaje al aumentar la temperatura y cálculo de la energía gap del material, siendo el valor obtenido es un 19\% superior al del material puro.
