\documentclass[11pt]{article}

    \usepackage[breakable]{tcolorbox}
    \usepackage{parskip} % Stop auto-indenting (to mimic markdown behaviour)
    
    \usepackage{iftex}
    \ifPDFTeX
    	\usepackage[T1]{fontenc}
    	\usepackage{mathpazo}
    \else
    	\usepackage{fontspec}
    \fi

    % Basic figure setup, for now with no caption control since it's done
    % automatically by Pandoc (which extracts ![](path) syntax from Markdown).
    \usepackage{graphicx}
    % Maintain compatibility with old templates. Remove in nbconvert 6.0
    \let\Oldincludegraphics\includegraphics
    % Ensure that by default, figures have no caption (until we provide a
    % proper Figure object with a Caption API and a way to capture that
    % in the conversion process - todo).
    \usepackage{caption}
    \DeclareCaptionFormat{nocaption}{}
    \captionsetup{format=nocaption,aboveskip=0pt,belowskip=0pt}

    \usepackage{float}
    \floatplacement{figure}{H} % forces figures to be placed at the correct location
    \usepackage{xcolor} % Allow colors to be defined
    \usepackage{enumerate} % Needed for markdown enumerations to work
    \usepackage{geometry} % Used to adjust the document margins
    \usepackage{amsmath} % Equations
    \usepackage{amssymb} % Equations
    \usepackage{textcomp} % defines textquotesingle
    % Hack from http://tex.stackexchange.com/a/47451/13684:
    \AtBeginDocument{%
        \def\PYZsq{\textquotesingle}% Upright quotes in Pygmentized code
    }
    \usepackage{upquote} % Upright quotes for verbatim code
    \usepackage{eurosym} % defines \euro
    \usepackage[mathletters]{ucs} % Extended unicode (utf-8) support
    \usepackage{fancyvrb} % verbatim replacement that allows latex
    \usepackage{grffile} % extends the file name processing of package graphics 
                         % to support a larger range
    \makeatletter % fix for old versions of grffile with XeLaTeX
    \@ifpackagelater{grffile}{2019/11/01}
    {
      % Do nothing on new versions
    }
    {
      \def\Gread@@xetex#1{%
        \IfFileExists{"\Gin@base".bb}%
        {\Gread@eps{\Gin@base.bb}}%
        {\Gread@@xetex@aux#1}%
      }
    }
    \makeatother
    \usepackage[Export]{adjustbox} % Used to constrain images to a maximum size
    \adjustboxset{max size={0.9\linewidth}{0.9\paperheight}}

    % The hyperref package gives us a pdf with properly built
    % internal navigation ('pdf bookmarks' for the table of contents,
    % internal cross-reference links, web links for URLs, etc.)
    \usepackage{hyperref}
    % The default LaTeX title has an obnoxious amount of whitespace. By default,
    % titling removes some of it. It also provides customization options.
    \usepackage{titling}
    \usepackage{longtable} % longtable support required by pandoc >1.10
    \usepackage{booktabs}  % table support for pandoc > 1.12.2
    \usepackage[inline]{enumitem} % IRkernel/repr support (it uses the enumerate* environment)
    \usepackage[normalem]{ulem} % ulem is needed to support strikethroughs (\sout)
                                % normalem makes italics be italics, not underlines
    \usepackage{mathrsfs}
    

    
    % Colors for the hyperref package
    \definecolor{urlcolor}{rgb}{0,.145,.698}
    \definecolor{linkcolor}{rgb}{.71,0.21,0.01}
    \definecolor{citecolor}{rgb}{.12,.54,.11}

    % ANSI colors
    \definecolor{ansi-black}{HTML}{3E424D}
    \definecolor{ansi-black-intense}{HTML}{282C36}
    \definecolor{ansi-red}{HTML}{E75C58}
    \definecolor{ansi-red-intense}{HTML}{B22B31}
    \definecolor{ansi-green}{HTML}{00A250}
    \definecolor{ansi-green-intense}{HTML}{007427}
    \definecolor{ansi-yellow}{HTML}{DDB62B}
    \definecolor{ansi-yellow-intense}{HTML}{B27D12}
    \definecolor{ansi-blue}{HTML}{208FFB}
    \definecolor{ansi-blue-intense}{HTML}{0065CA}
    \definecolor{ansi-magenta}{HTML}{D160C4}
    \definecolor{ansi-magenta-intense}{HTML}{A03196}
    \definecolor{ansi-cyan}{HTML}{60C6C8}
    \definecolor{ansi-cyan-intense}{HTML}{258F8F}
    \definecolor{ansi-white}{HTML}{C5C1B4}
    \definecolor{ansi-white-intense}{HTML}{A1A6B2}
    \definecolor{ansi-default-inverse-fg}{HTML}{FFFFFF}
    \definecolor{ansi-default-inverse-bg}{HTML}{000000}

    % common color for the border for error outputs.
    \definecolor{outerrorbackground}{HTML}{FFDFDF}

    % commands and environments needed by pandoc snippets
    % extracted from the output of `pandoc -s`
    \providecommand{\tightlist}{%
      \setlength{\itemsep}{0pt}\setlength{\parskip}{0pt}}
    \DefineVerbatimEnvironment{Highlighting}{Verbatim}{commandchars=\\\{\}}
    % Add ',fontsize=\small' for more characters per line
    \newenvironment{Shaded}{}{}
    \newcommand{\KeywordTok}[1]{\textcolor[rgb]{0.00,0.44,0.13}{\textbf{{#1}}}}
    \newcommand{\DataTypeTok}[1]{\textcolor[rgb]{0.56,0.13,0.00}{{#1}}}
    \newcommand{\DecValTok}[1]{\textcolor[rgb]{0.25,0.63,0.44}{{#1}}}
    \newcommand{\BaseNTok}[1]{\textcolor[rgb]{0.25,0.63,0.44}{{#1}}}
    \newcommand{\FloatTok}[1]{\textcolor[rgb]{0.25,0.63,0.44}{{#1}}}
    \newcommand{\CharTok}[1]{\textcolor[rgb]{0.25,0.44,0.63}{{#1}}}
    \newcommand{\StringTok}[1]{\textcolor[rgb]{0.25,0.44,0.63}{{#1}}}
    \newcommand{\CommentTok}[1]{\textcolor[rgb]{0.38,0.63,0.69}{\textit{{#1}}}}
    \newcommand{\OtherTok}[1]{\textcolor[rgb]{0.00,0.44,0.13}{{#1}}}
    \newcommand{\AlertTok}[1]{\textcolor[rgb]{1.00,0.00,0.00}{\textbf{{#1}}}}
    \newcommand{\FunctionTok}[1]{\textcolor[rgb]{0.02,0.16,0.49}{{#1}}}
    \newcommand{\RegionMarkerTok}[1]{{#1}}
    \newcommand{\ErrorTok}[1]{\textcolor[rgb]{1.00,0.00,0.00}{\textbf{{#1}}}}
    \newcommand{\NormalTok}[1]{{#1}}
    
    % Additional commands for more recent versions of Pandoc
    \newcommand{\ConstantTok}[1]{\textcolor[rgb]{0.53,0.00,0.00}{{#1}}}
    \newcommand{\SpecialCharTok}[1]{\textcolor[rgb]{0.25,0.44,0.63}{{#1}}}
    \newcommand{\VerbatimStringTok}[1]{\textcolor[rgb]{0.25,0.44,0.63}{{#1}}}
    \newcommand{\SpecialStringTok}[1]{\textcolor[rgb]{0.73,0.40,0.53}{{#1}}}
    \newcommand{\ImportTok}[1]{{#1}}
    \newcommand{\DocumentationTok}[1]{\textcolor[rgb]{0.73,0.13,0.13}{\textit{{#1}}}}
    \newcommand{\AnnotationTok}[1]{\textcolor[rgb]{0.38,0.63,0.69}{\textbf{\textit{{#1}}}}}
    \newcommand{\CommentVarTok}[1]{\textcolor[rgb]{0.38,0.63,0.69}{\textbf{\textit{{#1}}}}}
    \newcommand{\VariableTok}[1]{\textcolor[rgb]{0.10,0.09,0.49}{{#1}}}
    \newcommand{\ControlFlowTok}[1]{\textcolor[rgb]{0.00,0.44,0.13}{\textbf{{#1}}}}
    \newcommand{\OperatorTok}[1]{\textcolor[rgb]{0.40,0.40,0.40}{{#1}}}
    \newcommand{\BuiltInTok}[1]{{#1}}
    \newcommand{\ExtensionTok}[1]{{#1}}
    \newcommand{\PreprocessorTok}[1]{\textcolor[rgb]{0.74,0.48,0.00}{{#1}}}
    \newcommand{\AttributeTok}[1]{\textcolor[rgb]{0.49,0.56,0.16}{{#1}}}
    \newcommand{\InformationTok}[1]{\textcolor[rgb]{0.38,0.63,0.69}{\textbf{\textit{{#1}}}}}
    \newcommand{\WarningTok}[1]{\textcolor[rgb]{0.38,0.63,0.69}{\textbf{\textit{{#1}}}}}
    
    
    % Define a nice break command that doesn't care if a line doesn't already
    % exist.
    \def\br{\hspace*{\fill} \\* }
    % Math Jax compatibility definitions
    \def\gt{>}
    \def\lt{<}
    \let\Oldtex\TeX
    \let\Oldlatex\LaTeX
    \renewcommand{\TeX}{\textrm{\Oldtex}}
    \renewcommand{\LaTeX}{\textrm{\Oldlatex}}
    % Document parameters
    % Document title
    \title{UNED \\ Técnicas Experimentales IV \\ \large Física Nuclear}
    \author{Félix Rodríguez Lagonell}
    \date{2021 septiembre}
    
    
    
    
% Pygments definitions
\makeatletter
\def\PY@reset{\let\PY@it=\relax \let\PY@bf=\relax%
    \let\PY@ul=\relax \let\PY@tc=\relax%
    \let\PY@bc=\relax \let\PY@ff=\relax}
\def\PY@tok#1{\csname PY@tok@#1\endcsname}
\def\PY@toks#1+{\ifx\relax#1\empty\else%
    \PY@tok{#1}\expandafter\PY@toks\fi}
\def\PY@do#1{\PY@bc{\PY@tc{\PY@ul{%
    \PY@it{\PY@bf{\PY@ff{#1}}}}}}}
\def\PY#1#2{\PY@reset\PY@toks#1+\relax+\PY@do{#2}}

\@namedef{PY@tok@w}{\def\PY@tc##1{\textcolor[rgb]{0.73,0.73,0.73}{##1}}}
\@namedef{PY@tok@c}{\let\PY@it=\textit\def\PY@tc##1{\textcolor[rgb]{0.25,0.50,0.50}{##1}}}
\@namedef{PY@tok@cp}{\def\PY@tc##1{\textcolor[rgb]{0.74,0.48,0.00}{##1}}}
\@namedef{PY@tok@k}{\let\PY@bf=\textbf\def\PY@tc##1{\textcolor[rgb]{0.00,0.50,0.00}{##1}}}
\@namedef{PY@tok@kp}{\def\PY@tc##1{\textcolor[rgb]{0.00,0.50,0.00}{##1}}}
\@namedef{PY@tok@kt}{\def\PY@tc##1{\textcolor[rgb]{0.69,0.00,0.25}{##1}}}
\@namedef{PY@tok@o}{\def\PY@tc##1{\textcolor[rgb]{0.40,0.40,0.40}{##1}}}
\@namedef{PY@tok@ow}{\let\PY@bf=\textbf\def\PY@tc##1{\textcolor[rgb]{0.67,0.13,1.00}{##1}}}
\@namedef{PY@tok@nb}{\def\PY@tc##1{\textcolor[rgb]{0.00,0.50,0.00}{##1}}}
\@namedef{PY@tok@nf}{\def\PY@tc##1{\textcolor[rgb]{0.00,0.00,1.00}{##1}}}
\@namedef{PY@tok@nc}{\let\PY@bf=\textbf\def\PY@tc##1{\textcolor[rgb]{0.00,0.00,1.00}{##1}}}
\@namedef{PY@tok@nn}{\let\PY@bf=\textbf\def\PY@tc##1{\textcolor[rgb]{0.00,0.00,1.00}{##1}}}
\@namedef{PY@tok@ne}{\let\PY@bf=\textbf\def\PY@tc##1{\textcolor[rgb]{0.82,0.25,0.23}{##1}}}
\@namedef{PY@tok@nv}{\def\PY@tc##1{\textcolor[rgb]{0.10,0.09,0.49}{##1}}}
\@namedef{PY@tok@no}{\def\PY@tc##1{\textcolor[rgb]{0.53,0.00,0.00}{##1}}}
\@namedef{PY@tok@nl}{\def\PY@tc##1{\textcolor[rgb]{0.63,0.63,0.00}{##1}}}
\@namedef{PY@tok@ni}{\let\PY@bf=\textbf\def\PY@tc##1{\textcolor[rgb]{0.60,0.60,0.60}{##1}}}
\@namedef{PY@tok@na}{\def\PY@tc##1{\textcolor[rgb]{0.49,0.56,0.16}{##1}}}
\@namedef{PY@tok@nt}{\let\PY@bf=\textbf\def\PY@tc##1{\textcolor[rgb]{0.00,0.50,0.00}{##1}}}
\@namedef{PY@tok@nd}{\def\PY@tc##1{\textcolor[rgb]{0.67,0.13,1.00}{##1}}}
\@namedef{PY@tok@s}{\def\PY@tc##1{\textcolor[rgb]{0.73,0.13,0.13}{##1}}}
\@namedef{PY@tok@sd}{\let\PY@it=\textit\def\PY@tc##1{\textcolor[rgb]{0.73,0.13,0.13}{##1}}}
\@namedef{PY@tok@si}{\let\PY@bf=\textbf\def\PY@tc##1{\textcolor[rgb]{0.73,0.40,0.53}{##1}}}
\@namedef{PY@tok@se}{\let\PY@bf=\textbf\def\PY@tc##1{\textcolor[rgb]{0.73,0.40,0.13}{##1}}}
\@namedef{PY@tok@sr}{\def\PY@tc##1{\textcolor[rgb]{0.73,0.40,0.53}{##1}}}
\@namedef{PY@tok@ss}{\def\PY@tc##1{\textcolor[rgb]{0.10,0.09,0.49}{##1}}}
\@namedef{PY@tok@sx}{\def\PY@tc##1{\textcolor[rgb]{0.00,0.50,0.00}{##1}}}
\@namedef{PY@tok@m}{\def\PY@tc##1{\textcolor[rgb]{0.40,0.40,0.40}{##1}}}
\@namedef{PY@tok@gh}{\let\PY@bf=\textbf\def\PY@tc##1{\textcolor[rgb]{0.00,0.00,0.50}{##1}}}
\@namedef{PY@tok@gu}{\let\PY@bf=\textbf\def\PY@tc##1{\textcolor[rgb]{0.50,0.00,0.50}{##1}}}
\@namedef{PY@tok@gd}{\def\PY@tc##1{\textcolor[rgb]{0.63,0.00,0.00}{##1}}}
\@namedef{PY@tok@gi}{\def\PY@tc##1{\textcolor[rgb]{0.00,0.63,0.00}{##1}}}
\@namedef{PY@tok@gr}{\def\PY@tc##1{\textcolor[rgb]{1.00,0.00,0.00}{##1}}}
\@namedef{PY@tok@ge}{\let\PY@it=\textit}
\@namedef{PY@tok@gs}{\let\PY@bf=\textbf}
\@namedef{PY@tok@gp}{\let\PY@bf=\textbf\def\PY@tc##1{\textcolor[rgb]{0.00,0.00,0.50}{##1}}}
\@namedef{PY@tok@go}{\def\PY@tc##1{\textcolor[rgb]{0.53,0.53,0.53}{##1}}}
\@namedef{PY@tok@gt}{\def\PY@tc##1{\textcolor[rgb]{0.00,0.27,0.87}{##1}}}
\@namedef{PY@tok@err}{\def\PY@bc##1{{\setlength{\fboxsep}{\string -\fboxrule}\fcolorbox[rgb]{1.00,0.00,0.00}{1,1,1}{\strut ##1}}}}
\@namedef{PY@tok@kc}{\let\PY@bf=\textbf\def\PY@tc##1{\textcolor[rgb]{0.00,0.50,0.00}{##1}}}
\@namedef{PY@tok@kd}{\let\PY@bf=\textbf\def\PY@tc##1{\textcolor[rgb]{0.00,0.50,0.00}{##1}}}
\@namedef{PY@tok@kn}{\let\PY@bf=\textbf\def\PY@tc##1{\textcolor[rgb]{0.00,0.50,0.00}{##1}}}
\@namedef{PY@tok@kr}{\let\PY@bf=\textbf\def\PY@tc##1{\textcolor[rgb]{0.00,0.50,0.00}{##1}}}
\@namedef{PY@tok@bp}{\def\PY@tc##1{\textcolor[rgb]{0.00,0.50,0.00}{##1}}}
\@namedef{PY@tok@fm}{\def\PY@tc##1{\textcolor[rgb]{0.00,0.00,1.00}{##1}}}
\@namedef{PY@tok@vc}{\def\PY@tc##1{\textcolor[rgb]{0.10,0.09,0.49}{##1}}}
\@namedef{PY@tok@vg}{\def\PY@tc##1{\textcolor[rgb]{0.10,0.09,0.49}{##1}}}
\@namedef{PY@tok@vi}{\def\PY@tc##1{\textcolor[rgb]{0.10,0.09,0.49}{##1}}}
\@namedef{PY@tok@vm}{\def\PY@tc##1{\textcolor[rgb]{0.10,0.09,0.49}{##1}}}
\@namedef{PY@tok@sa}{\def\PY@tc##1{\textcolor[rgb]{0.73,0.13,0.13}{##1}}}
\@namedef{PY@tok@sb}{\def\PY@tc##1{\textcolor[rgb]{0.73,0.13,0.13}{##1}}}
\@namedef{PY@tok@sc}{\def\PY@tc##1{\textcolor[rgb]{0.73,0.13,0.13}{##1}}}
\@namedef{PY@tok@dl}{\def\PY@tc##1{\textcolor[rgb]{0.73,0.13,0.13}{##1}}}
\@namedef{PY@tok@s2}{\def\PY@tc##1{\textcolor[rgb]{0.73,0.13,0.13}{##1}}}
\@namedef{PY@tok@sh}{\def\PY@tc##1{\textcolor[rgb]{0.73,0.13,0.13}{##1}}}
\@namedef{PY@tok@s1}{\def\PY@tc##1{\textcolor[rgb]{0.73,0.13,0.13}{##1}}}
\@namedef{PY@tok@mb}{\def\PY@tc##1{\textcolor[rgb]{0.40,0.40,0.40}{##1}}}
\@namedef{PY@tok@mf}{\def\PY@tc##1{\textcolor[rgb]{0.40,0.40,0.40}{##1}}}
\@namedef{PY@tok@mh}{\def\PY@tc##1{\textcolor[rgb]{0.40,0.40,0.40}{##1}}}
\@namedef{PY@tok@mi}{\def\PY@tc##1{\textcolor[rgb]{0.40,0.40,0.40}{##1}}}
\@namedef{PY@tok@il}{\def\PY@tc##1{\textcolor[rgb]{0.40,0.40,0.40}{##1}}}
\@namedef{PY@tok@mo}{\def\PY@tc##1{\textcolor[rgb]{0.40,0.40,0.40}{##1}}}
\@namedef{PY@tok@ch}{\let\PY@it=\textit\def\PY@tc##1{\textcolor[rgb]{0.25,0.50,0.50}{##1}}}
\@namedef{PY@tok@cm}{\let\PY@it=\textit\def\PY@tc##1{\textcolor[rgb]{0.25,0.50,0.50}{##1}}}
\@namedef{PY@tok@cpf}{\let\PY@it=\textit\def\PY@tc##1{\textcolor[rgb]{0.25,0.50,0.50}{##1}}}
\@namedef{PY@tok@c1}{\let\PY@it=\textit\def\PY@tc##1{\textcolor[rgb]{0.25,0.50,0.50}{##1}}}
\@namedef{PY@tok@cs}{\let\PY@it=\textit\def\PY@tc##1{\textcolor[rgb]{0.25,0.50,0.50}{##1}}}

\def\PYZbs{\char`\\}
\def\PYZus{\char`\_}
\def\PYZob{\char`\{}
\def\PYZcb{\char`\}}
\def\PYZca{\char`\^}
\def\PYZam{\char`\&}
\def\PYZlt{\char`\<}
\def\PYZgt{\char`\>}
\def\PYZsh{\char`\#}
\def\PYZpc{\char`\%}
\def\PYZdl{\char`\$}
\def\PYZhy{\char`\-}
\def\PYZsq{\char`\'}
\def\PYZdq{\char`\"}
\def\PYZti{\char`\~}
% for compatibility with earlier versions
\def\PYZat{@}
\def\PYZlb{[}
\def\PYZrb{]}
\makeatother


    % For linebreaks inside Verbatim environment from package fancyvrb. 
    \makeatletter
        \newbox\Wrappedcontinuationbox 
        \newbox\Wrappedvisiblespacebox 
        \newcommand*\Wrappedvisiblespace {\textcolor{red}{\textvisiblespace}} 
        \newcommand*\Wrappedcontinuationsymbol {\textcolor{red}{\llap{\tiny$\m@th\hookrightarrow$}}} 
        \newcommand*\Wrappedcontinuationindent {3ex } 
        \newcommand*\Wrappedafterbreak {\kern\Wrappedcontinuationindent\copy\Wrappedcontinuationbox} 
        % Take advantage of the already applied Pygments mark-up to insert 
        % potential linebreaks for TeX processing. 
        %        {, <, #, %, $, ' and ": go to next line. 
        %        _, }, ^, &, >, - and ~: stay at end of broken line. 
        % Use of \textquotesingle for straight quote. 
        \newcommand*\Wrappedbreaksatspecials {% 
            \def\PYGZus{\discretionary{\char`\_}{\Wrappedafterbreak}{\char`\_}}% 
            \def\PYGZob{\discretionary{}{\Wrappedafterbreak\char`\{}{\char`\{}}% 
            \def\PYGZcb{\discretionary{\char`\}}{\Wrappedafterbreak}{\char`\}}}% 
            \def\PYGZca{\discretionary{\char`\^}{\Wrappedafterbreak}{\char`\^}}% 
            \def\PYGZam{\discretionary{\char`\&}{\Wrappedafterbreak}{\char`\&}}% 
            \def\PYGZlt{\discretionary{}{\Wrappedafterbreak\char`\<}{\char`\<}}% 
            \def\PYGZgt{\discretionary{\char`\>}{\Wrappedafterbreak}{\char`\>}}% 
            \def\PYGZsh{\discretionary{}{\Wrappedafterbreak\char`\#}{\char`\#}}% 
            \def\PYGZpc{\discretionary{}{\Wrappedafterbreak\char`\%}{\char`\%}}% 
            \def\PYGZdl{\discretionary{}{\Wrappedafterbreak\char`\$}{\char`\$}}% 
            \def\PYGZhy{\discretionary{\char`\-}{\Wrappedafterbreak}{\char`\-}}% 
            \def\PYGZsq{\discretionary{}{\Wrappedafterbreak\textquotesingle}{\textquotesingle}}% 
            \def\PYGZdq{\discretionary{}{\Wrappedafterbreak\char`\"}{\char`\"}}% 
            \def\PYGZti{\discretionary{\char`\~}{\Wrappedafterbreak}{\char`\~}}% 
        } 
        % Some characters . , ; ? ! / are not pygmentized. 
        % This macro makes them "active" and they will insert potential linebreaks 
        \newcommand*\Wrappedbreaksatpunct {% 
            \lccode`\~`\.\lowercase{\def~}{\discretionary{\hbox{\char`\.}}{\Wrappedafterbreak}{\hbox{\char`\.}}}% 
            \lccode`\~`\,\lowercase{\def~}{\discretionary{\hbox{\char`\,}}{\Wrappedafterbreak}{\hbox{\char`\,}}}% 
            \lccode`\~`\;\lowercase{\def~}{\discretionary{\hbox{\char`\;}}{\Wrappedafterbreak}{\hbox{\char`\;}}}% 
            \lccode`\~`\:\lowercase{\def~}{\discretionary{\hbox{\char`\:}}{\Wrappedafterbreak}{\hbox{\char`\:}}}% 
            \lccode`\~`\?\lowercase{\def~}{\discretionary{\hbox{\char`\?}}{\Wrappedafterbreak}{\hbox{\char`\?}}}% 
            \lccode`\~`\!\lowercase{\def~}{\discretionary{\hbox{\char`\!}}{\Wrappedafterbreak}{\hbox{\char`\!}}}% 
            \lccode`\~`\/\lowercase{\def~}{\discretionary{\hbox{\char`\/}}{\Wrappedafterbreak}{\hbox{\char`\/}}}% 
            \catcode`\.\active
            \catcode`\,\active 
            \catcode`\;\active
            \catcode`\:\active
            \catcode`\?\active
            \catcode`\!\active
            \catcode`\/\active 
            \lccode`\~`\~ 	
        }
    \makeatother

    \let\OriginalVerbatim=\Verbatim
    \makeatletter
    \renewcommand{\Verbatim}[1][1]{%
        %\parskip\z@skip
        \sbox\Wrappedcontinuationbox {\Wrappedcontinuationsymbol}%
        \sbox\Wrappedvisiblespacebox {\FV@SetupFont\Wrappedvisiblespace}%
        \def\FancyVerbFormatLine ##1{\hsize\linewidth
            \vtop{\raggedright\hyphenpenalty\z@\exhyphenpenalty\z@
                \doublehyphendemerits\z@\finalhyphendemerits\z@
                \strut ##1\strut}%
        }%
        % If the linebreak is at a space, the latter will be displayed as visible
        % space at end of first line, and a continuation symbol starts next line.
        % Stretch/shrink are however usually zero for typewriter font.
        \def\FV@Space {%
            \nobreak\hskip\z@ plus\fontdimen3\font minus\fontdimen4\font
            \discretionary{\copy\Wrappedvisiblespacebox}{\Wrappedafterbreak}
            {\kern\fontdimen2\font}%
        }%
        
        % Allow breaks at special characters using \PYG... macros.
        \Wrappedbreaksatspecials
        % Breaks at punctuation characters . , ; ? ! and / need catcode=\active 	
        \OriginalVerbatim[#1,codes*=\Wrappedbreaksatpunct]%
    }
    \makeatother

    % Exact colors from NB
    \definecolor{incolor}{HTML}{303F9F}
    \definecolor{outcolor}{HTML}{D84315}
    \definecolor{cellborder}{HTML}{CFCFCF}
    \definecolor{cellbackground}{HTML}{F7F7F7}
    
    % prompt
    \makeatletter
    \newcommand{\boxspacing}{\kern\kvtcb@left@rule\kern\kvtcb@boxsep}
    \makeatother
    \newcommand{\prompt}[4]{
        {\ttfamily\llap{{\color{#2}[#3]:\hspace{3pt}#4}}\vspace{-\baselineskip}}
    }
    

    
    % Prevent overflowing lines due to hard-to-break entities
    \sloppy 
    % Setup hyperref package
    \hypersetup{
      breaklinks=true,  % so long urls are correctly broken across lines
      colorlinks=true,
      urlcolor=urlcolor,
      linkcolor=linkcolor,
      citecolor=citecolor,
      }
    % Slightly bigger margins than the latex defaults
    
    \geometry{verbose,tmargin=1in,bmargin=1in,lmargin=1in,rmargin=1in}
    
    

\begin{document}
    
    \maketitle
    
    

    
    \hypertarget{caracterizaciuxf3n-de-un-contador-geiger-muller}{%
\section{Caracterización de un contador
Geiger-Muller}\label{caracterizaciuxf3n-de-un-contador-geiger-muller}}

Se realizan una serie de medidas en el laboratorio con la idea de hacer
la caracterización de un contador Geiger-Muller (G-M).

    \hypertarget{curva-caracteruxedstica}{%
\subsection{Curva característica}\label{curva-caracteruxedstica}}

En primer lugar se mide la actividad de una muestra radiactiva de
cobalto 60 \(Co^{60}\) y estroncio 90 \(Sr^{90}\) variando la tensión de
entrada.

    \begin{Verbatim}[commandchars=\\\{\}]
    |   tension (V) |   cuentas |        bq
----+---------------+-----------+-----------
  0 |           360 |        85 |  0.944444
  1 |           400 |      2099 | 23.3222
  2 |           440 |      4232 | 47.0222
  3 |           480 |      4187 | 46.5222
  4 |           520 |      4352 | 48.3556
  5 |           560 |      4265 | 47.3889
  6 |           600 |      4260 | 47.3333
  7 |           640 |      4393 | 48.8111
  8 |           680 |      4438 | 49.3111
  9 |           720 |      4451 | 49.4556
 10 |           760 |      4508 | 50.0889
 11 |           800 |      4724 | 52.4889
 12 |           840 |      4914 | 54.6
 13 |           880 |      5292 | 58.8
    \end{Verbatim}

    \begin{center}
    \adjustimage{max size={0.9\linewidth}{0.9\paperheight}}{Nuclear_files/Nuclear_4_0.png}
    \end{center}
    { \hspace*{\fill} \\}
     
            
    
    Por inspección visual podemos ver que la tensión umbral viene dada por
el valor \(V_u=440V\). La longitud de la meseta en este caso sería
\(V_f - V_i = 760 - 480 = 280V\), siendo \(V_f\) y \(V_i\) la tensión al
comienzo y al final de la meseta. Puesto que la longitud de la meseta es
superior a 200V tenemos que la tensión de trabajo es
\(V_t = V_i + 100 =580V\)

    

    La pendiente relativa al punto medio de la meseta expresada en tanto por
ciento por 100V viene data por la expresión \[
P = \frac{\frac{R_f - R_i}{(R_f + R_i)/2}}{(V_f - V_i)/100}x100
\]
 
            
    
    Siendo \(R_i = 46.52\), \(R_f=50.09\) los Bq al comienzo y final de la
meseta, respectivamente. Con estos datos obtenemos un valor de
\(P=2.64\) en \% (por 100V)

    

    \hypertarget{fondo}{%
\subsection{Fondo}\label{fondo}}

    Mediante el método de las dos fuentes podemos calcular el tiempo de
resolución del contador Geiger-Muller. Si se mide durante 90s la
actividad de una muestra de \(Co^{60}\) y \(Sr^{90}\), obtenemos la
siguiente tabla

    \begin{Verbatim}[commandchars=\\\{\}]
     |    0 |    1 |    2 |    3 |   media |    bq
-----+------+------+------+------+---------+-------
 A1  | 4036 | 4110 | 4096 | 4082 | 4081    | 45.34
 A12 | 7577 | 7545 | 7630 | 7433 | 7546.25 | 83.85
 A2  | 3750 | 3806 | 3833 | 3819 | 3802    | 42.24
 F   |   27 |   14 |   30 |   29 |   25    |  0.28
    \end{Verbatim}

    Donde \(A_1\) corresponde a la actividad de la muestra de cobalto,
\(A_2\) a la muestra de estroncio, \(A_{12}\) la actividad de ambas
muestras y \(F\) la radiación de fondo (sin muestra). Con estas medidas
podemos calcular el tiempo de resolución del contador mediante la
siguiente fórmula

\[
\tau = \frac{A_{12} + F - A_1 - A_2}{A_1^2 + A_2^2 - F^2 - A_{12}^2}
\]
 
            
    
    Sustituyendo valores obtenemos que el tiempo de resolución es
\(0.0011s\)

    

    \hypertarget{eficiencia-del-detector}{%
\subsection{Eficiencia del detector}\label{eficiencia-del-detector}}

    \textbf{Fuente: \(Co^{60}\)}

\begin{itemize}
\item
  Tipo de emisión: emisión \(\beta^{-}\) y \(\gamma\)
\item
  Tiempo de medida: \(t=90s\)
\item
  Número de cuentas: \(L=4537\)
\item
  Fondo: \(F = 25\)
\item
  Tasa de recuento neta: \(A'= \frac{(L-F)}{t} = 50.13 Bq\)
\item
  Actividad inicial de la muestra: \(A_0 = 1 \mu Ci = 37000 Bq\)
\item
  Fecha: abril 2010. Tiempo transcurrido hasta marzo 2012 cuando fueron
  tomadas las medidas: \(T_m = 23 meses (1.92 a)\)
\item
  Período:
  \(T_{1/2} = 5.26 a -> \lambda = \frac{ln2}{T_{1/2}} = 0.13 a^{-1}\)
\item
  Actividad corregida: \(A = A_0e^{- \lambda T_m} = 28728 Bq\)
\item
  Eficiencia \(\epsilon = A'/A = 0.0018\)
\end{itemize}

    \textbf{Fuente: \(Sr^{90}\)}

\begin{itemize}
\item
  Tipo de emisión: emisión \(\beta^{-}\)
\item
  Tiempo de medida: \(t=90s\)
\item
  Número de cuentas: \(L=34212\)
\item
  Fondo: \(F = 25\)
\item
  Tasa de recuento neta: \(A'= \frac{(L-F)}{t} = 380.13 Bq\)
\item
  Actividad inicial de la muestra: \(A_0 = 0.1 \mu Ci = 3700 Bq\)
\item
  Fecha: mayo 2010. Tiempo transcurrido hasta marzo 2012 cuando fueron
  tomadas las medidas: \(T_m = 22 meses (1.83 a)\)
\item
  Período:
  \(T_{1/2} = 28.5 a -> \lambda = \frac{ln2}{T_{1/2}} = 0.024 a^{-1}\)
\item
  Actividad corregida: \(A = A_0e^{- \lambda T_m} = 3538.93 Bq\)
\item
  Eficiencia \(\epsilon = A'/A = 0.11\)
\end{itemize}

    \hypertarget{conclusiones}{%
\subsection{Conclusiones}\label{conclusiones}}

La construcción de la curva característica de un detector Geiger-Muller
permite determinar el potencial óptimo de trabajo del dispositivo.

La pendiente de la meseta de un detector Geiger ordinario suele estar
entorno al 2\%-3\%, resultado que concuerda en gran medida con el valor
hallado en este experimento (2.64\%).

En cuanto al tiempo de respuesta de nuestro detector, podemos decir que
es de respuesta lenta \(\tau = 0.0011s\), ya que existen detectores con
un tiempo de resolución del microsegundo.

Es importante recalcar que conservar la posición de las muestras a lo
largo del experimento es importante puesto que la eficiencia del
detector, en especial la eficiencia geométrica, depende de la distancia
a la fuente de partículas. En cuanto a la eficiencia según el tipo de
fuente emisora, tenemos dos casos por analizar:

\begin{itemize}
\item
  Radiación gamma \(\gamma\): al ser radiación de alta energía tenemos
  que la mayoría de los fotones emitidos atraviesan el contador con
  mínima interacción. Cabe preguntarse si nuestro detector cumple los
  estándares habituales. Dado que los detectores G-M para la radiación
  gamma suelen tener una eficiencia de entorno al 1\% (o incluso menor)
  podemos concluir que nuestro detector sigue dicha normalidad, pues el
  valor obtenido es de entorno al \(\epsilon = 0.18\) \%
\item
  Radiación beta \(\beta\): los contadores G-M se caracterizan por ser
  buenos detectores de este tipo de radiación (corto alcance), por lo
  que suelen tener eficiencias muy cercanas al 100\%, sin embargo,
  nuestro detector G-M ofrece una eficiencia de entorno al 11\%, con lo
  que no podemos concluir que sea especialmente bueno para este tipo de
  radiación (al contrario que con la radiación gamma).
\end{itemize}

Por ahora hemos analizado la eficiencia del detector G-M según el tipo
de radiación, pero es importante aclarar que la eficiencia absoluta no
solo depende de este factor sino que también de factores geométricos
(como ya hemos comentado anteriormente sobre la posición de las
partículas). La relación entre las distintas eficiencias que puede
resumirse según la fórmula
\(\epsilon_t = \epsilon_{geo} x \epsilon_{i}\), y aquí intervienen
factores que pueden ser muy determinantes como son la energía de la
radiación, la probabilidad de interacción en el detector, el tiempo de
respuesta del detector, sus materiales de fabricación, su tamaño, etc.

Con todo lo dicho anteriormente, y teniendo en cueta las medidas tomadas
y los cálculos realizados, podemos llegar a la conclusión de que hemos
realizado una correcta caracterización de nuestro detector G-M.

    \hypertarget{estaduxedstica-aplicada-a-medidas-nucleares}{%
\section{Estadística aplicada a medidas
nucleares}\label{estaduxedstica-aplicada-a-medidas-nucleares}}

Usaremos el detector G-M de la práctica anterior para medir la actividad
de una muestra de \(Sr^{90}\). Los datos recogidos son los siguientes

    \begin{Verbatim}[commandchars=\\\{\}]
   medida |   cuentas
----------+-----------
        1 |      3736
        2 |      3847
        3 |      3758
        4 |      3706
        5 |      3785
        6 |      3802
        7 |      3708
        8 |      3761
        9 |      3779
       10 |      3800
       11 |      3704
       12 |      3812
       13 |      3685
       14 |      3760
       15 |      3641
       16 |      3703
       17 |      3541
       18 |      3746
       19 |      3659
       20 |      3581
       21 |      3735
       22 |      3794
       23 |      3684
       24 |      3748
       25 |      3793
       26 |      3776
       27 |      3721
       28 |      3766
       29 |      3809
       30 |      3735
    \end{Verbatim}
 
            
    
    Tras realizar \(N=30\) medidas obtenemos una media de \(n=3735.83\)
cuentas, que según la distribución de Poisson, corresponde una
desviaicón típica teórica de \(\sigma_t = 61.12\) cuentas. La desviación
típica experimental para nuestros datos es de \(\sigma_e = 67.85\). Por
lo tanto tenemos que las desviaciones típicas son muy similares, siendo
la experimental superior en torno al 12\%, además, podemos ver que
comparten la primera cifra significativa.

    

    Conocidas las desviaciones típicas podemos aplicar la fórmula

\[
\chi^2 = \frac{\sigma_e^2}{\sigma_t^2}(N-1)
\]

Para el cálculo de la \(\chi^2\) de Pearson
 
            
    
    \(\chi^2 = 35.74\)

    
 
            
    
    Teniendo en cuenta el valor anterior e interpolando valores de una tabla
de probabilidades según el valor de los grados de libertad \(f=29\),
podemos decir que la probabilidad es \(p=0.22\)

    

    \hypertarget{conclusiones}{%
\subsection{Conclusiones}\label{conclusiones}}

En cuanto a la calidad de nuestro detector podemos concluir que, dado
que la probabilidad se encuentra entre los valores \(0.10<p<0.9\) y
siendo las desviaciones típicas muy similares, nos encontramos ante un
detector fiable y aceptable, estando en concordancia con los criterios
de aceptación establecidos.

    \hypertarget{absorciuxf3n-de-partuxedculas-beta}{%
\section{Absorción de partículas
beta}\label{absorciuxf3n-de-partuxedculas-beta}}

Nuevamente usaremos el detector G-M y la una muestra de \(Sr^{90}\) de
las anteriores prácticas. Los datos recogidos interponiendo blindajes de
aluminio entre muestra y detector son los siguientes

    \begin{Verbatim}[commandchars=\\\{\}]
    |   espesor (mm) |   cuentas |   espesor (g/cm2)
----+----------------+-----------+-------------------
  0 |           1.5  |      4952 |           0.4125
  1 |           1.85 |      2440 |           0.50875
  2 |           2.2  |      1162 |           0.605
  3 |           2.65 |       511 |           0.72875
  4 |           3    |       213 |           0.825
  5 |           3.1  |       191 |           0.8525
  6 |           3.3  |        77 |           0.9075
  7 |           3.6  |        42 |           0.99
  8 |           4.1  |        39 |           1.1275
  9 |           4.45 |        31 |           1.22375
 10 |           4.9  |        32 |           1.3475
 11 |           5.25 |        41 |           1.44375
    \end{Verbatim}

    Donde el espesor en \(g/cm^2\) se ha obtenido multiplicando el espesor
(en cm) por la densidad del aluminio \(\rho = 2.75 g/cm^2\).

La gráfica del número de cuentas frente al espesor, en escala
semilogarítmica, es la siguiente

    \begin{center}
    \adjustimage{max size={0.9\linewidth}{0.9\paperheight}}{Nuclear_files/Nuclear_29_0.png}
    \end{center}
    { \hspace*{\fill} \\}
     
            
    
    Por inspección visual podemos apreciar que el alcance tiene un valor de
\(0.99g/cm^2\)

    

    Disponemos de un set de datos que relaciona linealmente el alcance con
la energía máxima de las partículas \(\beta\) emitidas por diferentes
nucleidos

    \begin{Verbatim}[commandchars=\\\{\}]
Coeficientes: 1.910
Término independiente: 0.163
Mean Squared Error (MSE): 0.01
R2: 0.99
    \end{Verbatim}

    \begin{center}
    \adjustimage{max size={0.9\linewidth}{0.9\paperheight}}{Nuclear_files/Nuclear_32_1.png}
    \end{center}
    { \hspace*{\fill} \\}
     
            
    
    Con los resultados de la regresión lineal y el valor de alcance obtenido
experimenalmente podemos estimar el valor de la energía máxima para
nuestra muestra radiactiva de emisión beta es \(E_m = 2.05MeV\)

    

    Podemos comparar estos valores experimentales frente a los resultados
dados por las siguientes ecuaciones, conocidas como las ecuaciones de
Feather:

\[
E = 1.845D + 0.245 \quad si \quad 0.3<D<1.5 \quad ;  \quad \\
E = 1.918D^{0.72} \quad si \quad 0.012<D<1.3
\]
 
            
    
    Teniendo en cuenta que nuestra fuente de estroncio 90 tiene una energía
máxima teórica de \(E_t = 2.28 MeV\), llegamos al resultado de un
alcance teórico \(D = 1.1g/cm^2\) según la primera fórmula de Feather.

    

    \hypertarget{conclusiones}{%
\subsection{Conclusiones}\label{conclusiones}}

Los resultados obtenidos son comparables ya que todos comparten orden de
magnitud y primera cifra significativa. Sin embargo, la diferencia entre
ambos resultados puede tener varias causas. Algunas de ellas pueden ser
errores sistemáticos durante el experimento (debido a mala calibración
de los instrumentos, desplazamiento de las muestras, errores en la
medición manual), falta de rigurosidad en las estimaciones (la
estimación del alcance se ha hecho por inspección visual), falta de
datos (más datos supone mayor fiabilidad en experimentos de naturaleza
aleatoria), etc. Además, hay que tener en cuenta que estamos comparando
valores entre dos métodos que son propiamente experimentales, sin un
análisis de confianza no podemos concluir que los valores obtenidos son
inconsistentes.

    \hypertarget{detectores-de-centelleo.-absorciuxf3n-de-radiaciuxf3n-gamma}{%
\section{Detectores de centelleo. Absorción de radiación
gamma}\label{detectores-de-centelleo.-absorciuxf3n-de-radiaciuxf3n-gamma}}

Colocamos distintos materiales (aluminio, plomo, hierro) de diferentes
espesores entre una muestra radiactiva de \(Co^{60}\) y un detector de
centelleo. Representamos dichos datos en la siguiente gráfica junto con
las regresiones lineales correspondientes. Además presentaremos algunos
resultados interesantes requeridos en esta práctica

    \begin{Verbatim}[commandchars=\\\{\}]
ALUMINIO
Densidad 2.7 g/cm3
Contaje espesor cero N\_0 = 11091 cuentas
Contaje mitad N\_12 = 5546 cuentas
Espesor de semirreducción X\_12 = 79.54 mm
Coeficiente de absorción lineal mu = 0.087 cm-1
Coeficiente de absorción másico mu\_m = 0.032 cm2/g
Número de átomos por cm3: 6.02626116026962e+22
Sección eficaz sigma = 1.4461051514657685e-24 cm2

PLOMO
Densidad 11.34 g/cm3
Contaje espesor cero N\_0 = 9360 cuentas
Contaje mitad N\_12 = 4680 cuentas
Espesor de semirreducción X\_12 = 22.54 mm
Coeficiente de absorción lineal mu = 0.308 cm-1
Coeficiente de absorción másico mu\_m = 0.027 cm2/g
Número de átomos por cm3: 3.295901361891892e+22
Sección eficaz sigma = 9.332363419312933e-24 cm2

HIERRO
Densidad 7.874 g/cm3
Contaje espesor cero N\_0 = 10172 cuentas
Contaje mitad N\_12 = 5086 cuentas
Espesor de semirreducción X\_12 = 30.55 mm
Coeficiente de absorción lineal mu = 0.227 cm-1
Coeficiente de absorción másico mu\_m = 0.029 cm2/g
Número de átomos por cm3: 8.491062108378548e+22
Sección eficaz sigma = 2.6721994446750493e-24 cm2

    \end{Verbatim}

    \begin{center}
    \adjustimage{max size={0.9\linewidth}{0.9\paperheight}}{Nuclear_files/Nuclear_38_1.png}
    \end{center}
    { \hspace*{\fill} \\}
    
    Finalmente calcularemos la eficiencia del detector de centelleo

    \textbf{Fuente: \(Co^{60}\)}

\begin{itemize}
\item
  Tipo de emisión: emisión \(\beta^{-}\) y \(\gamma\)
\item
  Tiempo de medida: \(t=90s\)
\item
  Número de cuentas: \(L=164463\)
\item
  Fondo: \(F = 499\)
\item
  Tasa de recuento neta: \(A'= \frac{(L-F)}{t} = 1821.82 Bq\)
\item
  Actividad inicial de la muestra: \(A_0 = 1 \mu Ci = 37000 Bq\)
\item
  Fecha: septiembre 2010. Tiempo transcurrido hasta marzo 2012 cuando
  fueron tomadas las medidas: \(T_m = 18 meses (1.5 a)\)
\item
  Período:
  \(T_{1/2} = 5.26 a -> \lambda = \frac{ln2}{T_{1/2}} = 0.13 a^{-1}\)
\item
  Actividad corregida: \(A = A_0e^{- \lambda T_m} = 30363 Bq\)
\item
  Eficiencia \(\epsilon = A'/A = 0.0599\)
\end{itemize}

    \hypertarget{conclusiones}{%
\subsection{Conclusiones}\label{conclusiones}}

Para las energías de desintegración de la muestra de \(Co^{60}\), que
son 1.172 MeV y 1.333 MeV, hemos obtenidos los siguientes valores
teóricos por inspección visual de las gráficas aportadas:

\textbf{Teórico}

\begin{itemize}
\item
  Aluminio: \(\mu = 0.15cm^{-1}\) y \(\mu_m = 0.025 cm^2/g\)
\item
  Plomo: \(\mu = 0.6 cm^{-1}\) y \(\mu_m = 0.03 cm^2/g\)
\end{itemize}

Mientras que los valores experimentales han sido

\textbf{Experimental}

\begin{itemize}
\item
  Aluminio: \(\mu = 0.087cm^{-1}\) y \(\mu_m = 0.032 cm^2/g\)
\item
  Plomo: \(\mu = 0.308 cm^{-1}\) y \(\mu_m = 0.027 cm^2/g\)
\end{itemize}

Podemos comprobar que valores teóricos y experimentales son muy
similares para el coeficiente de absorción másico (al redondear
comparten primera cifra significativa), sin embargo hay notables
diferencias para el valor de la absorción lineal. Estas diferencias
pueden deberse a diferentes motivos:

\begin{itemize}
\item
  Falta de rigor en los valores teóricos, obtenidos por inspección
  visual de una gŕafica.
\item
  No se han establecido intervalos de confianza ni medidas de error, ya
  sean sistemáticos en la medida de espesores, o bien aleatorios
  referentes a la escasez de medidas. Además no se tienen en cuenta
  otras fuentes de radiación que también pueden llegar a nuestro
  detector.
\item
  Al ser una muestra radiativa de emisión \(\beta\), es posible que
  nuestro detector haya capturado parte de éstas partículas, así como
  los fotones originados por efecto Compton y/o en la aniquilación
  electrón-positrón que conlleva la creación de pares
\end{itemize}

La ley de atenuación exponencial de radiación gamma es válida al
tratarse de un haz colimado y monoenergético siempre que el material
atravesado sea de poco espesor y naturaleza homogénea, condiciones que
no podemos garantizar en nuestro experimento ya que

\begin{itemize}
\item
  El haz no ha sido colimado y contiene varias energías.
\item
  No sabemos la pureza y homogeneidad del material blindado.
\end{itemize}

Dicho esto, a pesar de no poder garantizar la ley de atenuación
exponencial, sí es cierto que los datos se ajustan linealmente en una
escala semilogarítmica, por lo que en primera aproximación podrían
relacionarse a un comportamiento exponencial.

    \hypertarget{determinaciuxf3n-experimental-del-peruxedodo-de-una-muestra-radiactiva}{%
\section{Determinación experimental del período de una muestra
radiactiva}\label{determinaciuxf3n-experimental-del-peruxedodo-de-una-muestra-radiactiva}}

El detector utilizado es un detector de centelleo similar al de la
práctica anterior. Se utiliza una muestra de período corto \(Ba^{137}\)
que se obtiene con un generador de isótopos que contiene \(Cs^{137}\) en
equilibrio con el \(Ba^{137}\)

    \begin{Verbatim}[commandchars=\\\{\}]
    |   tiempo (m) |   cuentas
----+--------------+-----------
  0 |            0 |     10479
  1 |            1 |      6934
  2 |            2 |      5998
  3 |            3 |      4169
  4 |            4 |      3374
  5 |            5 |      2636
  6 |            6 |      1680
  7 |            7 |      1537
  8 |            8 |      1152
  9 |            9 |       737
 10 |           10 |       665
 11 |           11 |       488
 12 |           12 |       436
    \end{Verbatim}

    \begin{center}
    \adjustimage{max size={0.9\linewidth}{0.9\paperheight}}{Nuclear_files/Nuclear_45_0.png}
    \end{center}
    { \hspace*{\fill} \\}
    
    Aplicando logaritmos a la ecuación de la actividad de una muestra
radiactiva tenemos

\[
A(t) = A_0e^{-\lambda t} \quad -> \quad Ln (A(t)) = -\lambda t +Ln(A_0)
\]

Así la contstante de desintegración será la pendiente de una regresión
lineal. Por otro lado el período de semidesintegración es

\[
T_{1/2} = \frac{Ln(2)}{\lambda}
\]

    \begin{Verbatim}[commandchars=\\\{\}]
Coeficientes: -0.005
Término independiente: 9.194
Mean Squared Error (MSE): 0.01
R2: 0.99
    \end{Verbatim}

    \begin{center}
    \adjustimage{max size={0.9\linewidth}{0.9\paperheight}}{Nuclear_files/Nuclear_47_1.png}
    \end{center}
    { \hspace*{\fill} \\}
     
            
    
    Por lo tanto tenemos que la constante es \(\lambda = 0.005\) 1/s

    
 
            
    
    Siendo el período de semidesintegración 138.63 s

    

    \hypertarget{conclusiones}{%
\subsection{Conclusiones}\label{conclusiones}}

Hemos comprobado que nuestros datos cumplen una ley de decaimiento
exponencial, calculando así la constante de desintegración y el
correspondiente período de semidesintegración. Al tener un período de
semidesintegración un valor entorno a 2.3 minutos no podemos tomar
medidas de tiempo mayores. Si midiésemos cada 5 minutos, por ejemplo,
transcurriría casi dos veces el período de semidesintegración entre
medidas, reduciéndose casi dos veces a la mitad la población de núcleos.


    % Add a bibliography block to the postdoc
    
    
    
\end{document}
